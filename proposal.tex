\documentclass[a4paper,12pt]{article}
\pdfoutput=1

\usepackage{graphicx}
\usepackage{parskip}
\usepackage{amsmath}
\usepackage{amsfonts}
\usepackage{amssymb}
\usepackage{amstext}
\usepackage{amsbsy}
\usepackage{amsopn}
\usepackage{amsthm}
\usepackage{amsxtra}
\usepackage{multirow}
\usepackage{array}
\usepackage{color}
\usepackage{hyperref}
\usepackage{cite}


\begin{document}
\vskip 1.5cm

A strong supportive evidence for a discovery could come from Rydberg-based precision measurements.
Rydberg-Rydberg transitions are subject to different systematics and are in general less affected by nuclear physics.
Therefore, one of the ingredients of our proposal is to develop strategies to formulate a complementary set of observables that are sensitive to the presence of new mass-dependent light mediators in Rydberg atoms.

---


Maybe do not use the word "quantum defects" in the very frist paragarph:
In the second line of experiments, Ofer Firstenberg's group will evaluate bounds on new physics from IS of Rydberg atoms.
Rydberg-Rydberg transitions are, to a larger extent, free of nuclear short-range effects.
The quantum defects, encapsulating the deviation from the Rydberg formula for high-$n$ levels, decrease with increasing orbital angular momentum $l$.
Thus the difference between the quantum defect of different isotopes could turn out to be theoretically predictable,in particular using relativistic multiconfiguration codes \cite{Grant:2013}, to a level better than that required to observe (or bound) new physics.
Thus King-plot linearity may not be necessary, and IS spectroscopy for only two isotopes, for example $^{85}$Rb and $^{87}$Rb \cite{Fortagh:2011pra}, could suffice.
If a King-plot linearity check is necessary, then species with more than two stable isotopes will be required.
We will explore this both theoretical and experimentally.

Survey beyond-hydrogen terms: claim that relativistic/spesific-mass shifts are negligible.
Review the ones that are not, in particular the dominant contribution is core polarizability.
Hamiltonian:
\begin{equation}
Dont through an error
\end{equation}
Discussion. $E_{n,l}= <\psi_nl|H|\psi_nl>$, where the leading term is the Rydebrg energies $(R/n^2)$ and each of the extra terms has its corresponsdant.. here??

Explain that in the Rydberg world, Quantum defects is the standrad formalism -- give the mathematical formula for quantum defects, and their rough dependence on n and l.
Connection between the Hamiltonian terms and te QD notations.
Then as an example, give the contribution of core pol. in terms of QD and in terms of noramlized energies $<r^{-4}>/<r^{-1}>$.

To simplify the following discussion, the analysis that follows is done solely on the binding energy $E_nl$, rather than on a transitions $E-E$.
Give excuses for that (if the upper state is much closer to ionization, e.g. n=20..., then its binding energy is much smaller, the derivatives are much smallel, etc.).
Furthermore, we are going to analyze the simplest scheme --- a measurement or a single energy level for only two isotopes -- to show already its huge potential.
Further analysis (more transitions, more isotopes, and King plot if all else fails) -- is kept for the grant project itself).
Finally, we provide here only the analysis for the mass-less case $m_??=0$, leaving the rest to the future.

So then, the isoptope shift is anlayzed in terms of derivative with resepct to A:
\begin{equation}
Hi
\end{equation}
Go over the terms? Say which ones are the more problematic. Perhaps gently discuss the notion of "big but known" vs "small but uncertain". Uncertainties... 

Maybe graph.

Then as an example, give typical numbers for the important properties of, say, n=10 l=10. Give also the lifetimes, etc. --- one and only place we actually discuss a specific state.

Concluding
--- further analyze the above case (collect all experimental and theory data to improve the estimations of uncertainties), generalize to massive case, generelize to more than two isoptopes.
-- the theoreitcal anlaysis will provide, by the mid-term period of this grant,
-- the fesiability and a proper experimental plan. Afterwards, we will try to perform first experiments
--- and it will take longer if we realize it requires a different isoptope than Rb.

Prospective:
blockade for direct measurement of IS as beats.
Figure: mapping of a two-isotope-two-state system to single-atom-three-state system (using blockade) and then the beating for a "storage and retreival" process ------ analogus to Ramsey spectroscopy woth two isotopes in the ion trap ("gate", "phonon-mediated interaaction", "entangled state") (in our case "dipole-mediated interaction"). Very uncertain -- espesically since high densities (required for efficient blockade) will influance the IS.



\begin{thebibliography}{2}

\bibitem{Grant:2013}
 P.~Jonsson, G.~Gaigalas, J.~Bieron, C.~Froese Fischer, I.P.~Grant
  Comp.~Phys.~Comm.~ {\bf 184}, 2197-2203 (2013).
\bibitem{Fortagh:2011pra}
  M.~Mack, F.~Karlewski, H.~Hattermann, S.~H\"ockh, F.~Jessen, D.~Cano, J.~Fort\'agh,
  Phys.\ Rev.\ A {\bf 83}, 052515 (2011).

\end{thebibliography}
\end{document}
